\section{Численные эксперименты}
\label{sec:Chapter4} \index{Chapter4}

% \subsection{Тест 1}
\par Тесты производились на интересующей нас задаче – линейной системе с многими 
правыми частями, возникающей при решении задачи электромагнитного рассеяния 
методом интегральных уравнений [STAVTSEV]. Порядок системы - 14144, всего правых частей - 722.
\begin{figure}[H]
    \centering
    \includegraphics[width=0.7\linewidth]{images/4.pdf}
    \caption{}
    \label{fig:4}
\end{figure} 
\par Первый тест демонстрирует, что метод из статьи [GUENNOUNI] не сходится с требуемой точностью, в то время как 
версия с улучшениями, описанными в главе \ref{sec:Chapter3}, сходится линейно без проблем. 
Эксперимент проводился в одинарной точности для четырех правых частей с номерами: 0, 90, 180, 270. Его результаты представлены
на рис.\ref{fig:4}
% \subsection{Тест 2}
\begin{figure}[H]
    \centering
    \includegraphics[width=0.7\linewidth]{images/acceleration_15_rhs.pdf}
    \caption{}
    \label{fig:acceleration_15}
\end{figure}
\par 15 правых частей, уменьшения числа итераций, считаем в двойной точности
\par Второй тест демонстрирует, что улучшения, описанные в главе \ref{sec:Chapter3} позволяют получить ускоренную сходимость
по сравнению с решением систем с каждой правой частью в отдельности
% \subsection{Тест 3}
\par более 30 правых частей, демонстрация отсутствия взрыва невязки

\newpage
