\section{Введение и постановка задачи}
\label{sec:Chapter0} \index{Chapter0}
В ряде приложений возникают большие линейные системы с многими правыми частями. такую задачу можно записать в блочном виде:\\
\begin{figure}
    \centering
    % \includesvg[width=0.5\linewidth]{images/system.svg}
    % \caption{Caption}
    \label{fig:system}
\end{figure}

$$AX=B,$$
где $A$ - $N\times N$ невырожденная разреженная матрица системы;
$B$ - $N\times s$ невырожденная матрица, столбцы - правые части; 
$X$ - $N\times s$ матрица, столбцы - решения для соответствующих правых частей. 
Также еще предполагаем, что $s\ll N$.
Часто для решения таких задач используют прямые методы, однако Крыловские методы круче, да.
\subsection{Преимущества блочных крыловских методов}
Высокая производительность на вычислительных системах за счет блочных операций,\\
Более быстрая сходимость, по сравнению с неблочными методами [DIANNE O'LEARY] \\
В задачах со структурированными системами (например МКЭ) БКМ не разрушают структуру,
в отличие от прямых методов.\\
Чрезвычайно большие системы, которые не помещаются целиком в оперативную память 
можно решать с помощью блочных крыловских методов\\

\subsection{Блочные Крыловские методы}


\newpage
