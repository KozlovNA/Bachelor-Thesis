\section{Модификация блочного симметричного метода квазиминимальных невязок}
\label{sec:bsqmr_mod} \index{bsqmr_mod}

\par Один из ключевых элементов блочного симметричного метода квазиминимальных невязок \cite{doi:10.1137/0917019}
является процесс Грамма-Шмидта с квазискалярным произведением. Далее будет представлена модификация 
этого алгоритма, использующая настоящее QR-разложение. Тогда откроется возможности для более устойчивых реализаций
QR-разложения и их библиотечным реализациям, а также к нескольким новым свойствам невязок.

\par Блочный симметричный процесс Ланцоша приводит к следующему матричному соотношению:
\begin{equation}
    \label{eq:bsqmr_AVeqT}
    A \begin{pmatrix}
        V_1 & ... & V_k & V_{k+1} 
    \end{pmatrix} = \begin{pmatrix}
        V_1 & ... & V_k & V_{k+1} 
    \end{pmatrix} \begin{pmatrix}
        \alpha_1 & \delta_1 & & & \\
        \beta_2 & \alpha_2 & \delta_2 & & \\
        & \beta_3 & \ddots & \ddots & \\
        & & \ddots & \alpha_{k-1} & \delta_{k-1} \\
        & & & \beta_k & \alpha_k \\
        & & & & \beta_{k+1}
    \end{pmatrix},
\end{equation} 
где $\delta_{i-1} = \beta_i^T$ в версии из статьи \cite{doi:10.1137/0917019}, в нашей
модификации же получится другой вид для этой матрицы коэффициентов. Из \eqref{eq:bsqmr_AVeqT} для $k$-го блока следует: 
\begin{equation}
    \label{eq:bsqmr_last_block}
    AV_k = V_{k-1}\delta_{k-1} + V_k \alpha_k + V_{k+1} \beta_{k+1}
\end{equation}

При построении базиса в блочном крыловском пространстве, требуется выпонение следующего свойства:
\begin{equation}
    \label{eq:VTVeq0}
V_i^TV_j=0,\;i \neq j
\end{equation}  

Домножая слева выражение \eqref{eq:bsqmr_last_block} на $V_{k-1}^T$ и используя соотношение
\eqref{eq:VTVeq0} получаем системы линейных уравнений на матрицу $\delta_{k-1}$:
\begin{equation}
    \label{eq:delta_system}
    V_{k-1}^T V_{k-1} \delta_{k-1} = V_{k-1}^T A V_k.
\end{equation}

Сделав замену в \eqref{eq:bsqmr_last_block} вида $k \rightarrow k-1$ и учтя выражение 
\eqref{eq:delta_system} выразим $\delta_{k-1}$ через $\beta_k$:
\begin{equation*}
    V_{k-1}^T V_{k-1} \delta_{k-1} = \beta_{k}^T V_k^T V_k.
\end{equation*}
Введем обозначение $\gamma_k = V_k^T V_k$.

Тогда окончательный вид для $\delta_{k-1}$:
\begin{equation}
    \label{eq:delta_final}
    \delta_{k-1} = \gamma_{k-1}^{-1} \beta_k^T \gamma_k.
\end{equation}

Аналогично $\delta_{k-1}$ из \eqref{eq:bsqmr_last_block} получим системы линейных
уравнений на $\alpha_k$:
\begin{equation*}
    \gamma_k \alpha_k = V_k^T A V_k.
\end{equation*}
И воспользовавшись свойством \eqref{eq:VTVeq0} преобразуем выражение для $\alpha_k$:
\begin{equation}
    \alpha_k = \gamma_k^{-1} V_k^T (A V_k - V_{k-1} \delta_{k-1}).
\end{equation}

Выбор $\beta_{k+1}$ является произвольным и определяется целями исследователя, в 
предлагаемой модификации $\beta_{k+1}$ выбрано таким, чтобы выполнялось соотношение
$V_{k+1}^* V_{k+1} = I$, где $I$ - единичная $s \times s$ матрица. Этого можно достичь с помощью QR-разложения:
\begin{equation}
    V_{k+1}, \beta_{k+1} \xleftarrow{QR} A V_k - V_{k-1} \delta_{k-1} - V_k \alpha_k. 
\end{equation} 
Этот выбор обладает рядом преимуществ: \begin{enumerate}
    \item получение QR-разложения в сравнении с квази-QR-разложением является более устойчивой операцией, 
    \item на первой итерации алгоритм ведёт себя как обобщённый метод минимальных невязок, что обеспечивает на первой итерации достижение точного минимума невязки в построенном к этому моменту пространстве Крылова, что в свою очередь предотвращает большие скачки невязки на первых итерациях, как это наблюдается в алгоритме из статьи \cite{doi:10.1137/0917019}.
\end{enumerate}  

Но предлагаемый алгоритм обладает и рядом недостатков:
\begin{enumerate}
    \item В задаче электромагнитного рассеяния на миндалевидном теле, дискретизированном методом интегральных уравнений \cite{stavtsev2009application} метод не сходится до требуемого порога в арифметике с одинарной точностью. 
\end{enumerate}

Окончательный вид алгоритма:
\begin{algorithm}
    \caption{Модифицированный блочный симметричный метод квазиминимальных невязок}\label{alg:bsqmr_mod}
    \begin{algorithmic}
        \State $V_0 = P_{0} = P_{-1} = 0_{N \times s}$, $N$ - размер матрицы $A$, $s$ - количество правых частей.
        \State $c_0 = b_{-1} = b_0 = 0_{s \times s}$
        \State $a_0 = d_{-1} = d_0 = I_{s \times s}$
        \State $R_0 = B - AX_0$
        \State $V_1,\; \beta_1 \xleftarrow{QR} R_0 $
        \State $\gamma_{0} = I_{s \times s}$
        \State $\gamma_{1} = V_1^T V_1$
        % {\color{red}
        % \State $\omega_{-1} = 0_{m \times m}$
        % \State $\omega_0 = I$
        % }
        \State $\tilde{\tau}_1 = \beta_1 $
        \For {$k = 1, ... $}
            \State $\delta_{k-1} = \gamma_{k-1}^{-1} \beta_k^T \gamma_k $
            \State $\tilde{V}_{k+1} = AV_k - V_{k-1} \delta_{k-1}$
            \State $\alpha_k = \gamma_k^{-1} V_k^T \tilde{V}_{k+1}$
            \State $\tilde{V}_{k+1} = \tilde{V}_{k+1} - V_k \alpha_k$
            \State $ V_{k+1},\; \beta_{k+1} \xleftarrow{QR} \tilde{V}_{k+1} $
            \State $\gamma_{k+1} = V_{k+1}^T V_{k+1}$
            % \State {\color{red} $\omega_{k+1} = I$}
            \State $\theta_k = b_{k-2} \delta_{k-1}$
            \State $\eta_k = a_{k-1}d_{k-2}\delta_{k-1} + b_{k-1}\alpha_k$
            \State $\tilde{\zeta}_k = c_{k-1} d_{k-2} \delta_{k-1} + d_{k-1} \alpha_k$
            \State $ Q_k ,\; 
                    \begin{pmatrix} 
                        \zeta_k \\
                        0_{s \times s}
                    \end{pmatrix} \xleftarrow{QR} \begin{pmatrix}
                                        \tilde{\zeta}_k \\
                                        \omega_{k+1} \beta_{k+1}
                                     \end{pmatrix}$
            \State $\begin{pmatrix}
                        a_k & b_k \\
                        c_k & d_k
                    \end{pmatrix} \gets Q_k^*$
            \State $V_k = (V_k - P_{k-1}\eta_k - P_{k-2} \theta_k)\zeta_k^{-1}$
            \State $\tau_k = a_k \tilde{\tau}_k$
            \State $X_k = X_{k-1} + P_{k} \tau_{k}$
            \State $\tilde{\tau}_{k+1} = c_k \tilde{\tau}_k$
        \EndFor
    \end{algorithmic}
\end{algorithm}


\newpage
   