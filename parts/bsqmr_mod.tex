\section{Модификация блочного симметричного метода квазиминимальных невязок}
\label{sec:bsqmr_mod} \index{bsqmr_mod}

\par Один из главных элементов блочного симметричного метода квазиминимальных невязок \cite{doi:10.1137/0917019}
является процесс Грамма-Шмидта с квазискалярным произведением. Далее будет представлена модификация 
этого алгоритма, использующая процесс Грамма-Шмидта с настоящим скалярным произведением (QR-разложение), но
приводящая к тем же рекурентным формулам. Тогда откроется простор для более устойчивых реализаций
QR-разложения и их библиотечным имплементациям, а также к нескольким новым свойствам невязок.

\par Симметричный процесс Ланцоша приводит к следующему матричному соотношению:
\begin{equation}
    \label{eq:bsqmr_AVeqT}
    A \begin{pmatrix}
        V_1 & ... & V_k & V_{k+1} 
    \end{pmatrix} = \begin{pmatrix}
        V_1 & ... & V_k & V_{k+1} 
    \end{pmatrix} \begin{pmatrix}
        \alpha_1 & \delta_1 & & & \\
        \beta_2 & \alpha_2 & \delta_2 & & \\
        & \beta_3 & \ddots & \ddots & \\
        & & \ddots & \alpha_{m-1} & \delta_{m-1} \\
        & & & \beta_m & \alpha_m \\
        & & & & \beta_{m+1}
    \end{pmatrix},
\end{equation} 
где $\delta_{i-1} = \beta_i^T$ в версии из статье \cite{doi:10.1137/0917019}, в нашей
модификации же получится другой вид для этой матрицы коэффициентов. Из соотношения 
\eqref{eq:bsqmr_AVeqT}, выделяя последний блок, получаем 
\begin{equation}
    AV_k = V_{k-1}delta
\end{equation}

В ходе итераций поддерживается соотношение $V_i^TV_j=0,\;i \neq j$. 

\newpage
   