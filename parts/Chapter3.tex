\section{Улучшения блочного метода стабилизированных бисопряенных градиентов}
\label{sec:Chapter3} \index{Chapter3}
В данной главе предложены изменения, направленные на улучшение стабильности блочного метода 
бисопряженных градиентов. 
\subsection{Реортогонализация для поддержания биортогональных соотношений}
Для построения базиса в крылоском пространстве и построения невязок алгоритм 
строится таким образом, чтобы поддерживать следующие соотношения ортогональности:
\begin{equation}
    \label{eq:ort_alpha_poly}
    C(\mathbf{Q}_{k} \mathbf{R}_{k+1}) = 0,
\end{equation}
\begin{equation}
    \label{eq:ort_beta_poly}
    C^{(1)}(\mathbf{Q}_{k} \mathbf{P}_k) = 0. 
\end{equation}
Для построения процедуры реортогонализации эти полиномиальные соотношения необходимо 
перевести в матричный вид.
Используя полиномиальное соотношение для $\mathbf{R}_{k+1}$, получаем:
$$\mathbf{Q}_{k}\mathbf{R}_{k+1} = \mathbf{Q}_{k}\mathbf{R}_{k} - t\mathbf{Q}_{k}\mathbf{P}_{k}\alpha_k \implies S_k = R_k - AP_k\alpha_k $$
Тогда выражение \eqref{eq:ort_alpha_poly} можно представить в виде:
\begin{equation}
    \label{eq:ort_alpha_matr}
    \tilde{R}_0^* S_k = 0.
\end{equation}
В точной арифметике это соотношение выполняется строго, однако при вычислениях на компьютере 
соотношение \eqref{eq:ort_alpha_matr} выполняется с какой-то погрешностью. Для существенного уменьшения 
этой погрешности можно произвести ортогонализацию еще раз, взяв $S_k$ в качестве блока, к которому производится
ортогонализация:
\begin{equation}
    \label{eq:S_k_reort}
    S_k^{r} = S_k - AP_k\alpha_k^r.
\end{equation}
При этом мы стремимся поддерживать соотношение $\tilde{R}_0^*S_k^r=0$ с уточненным блоком $S_k^r$.
Тогда, домножая обе части выражения \eqref{eq:S_k_reort} слева на $\tilde{R}_0$, получим уравнение для 
поправки $\alpha_k^r$:
\begin{equation*}
    (\tilde{R}_0^TAP_k)\alpha_k^r = \tilde{R}_0^TS_k.
\end{equation*}

Поработаем теперь над выражением \eqref{eq:ort_beta_poly}. Используя полиномиальное соотношение для 
$P_{K+1}$, получаем следующее выражение:
\begin{equation}
    \label{eq:W_k_poly}
    t\mathbf{Q}_k\mathbf{P}_{k+1} = t\mathbf{Q}_k\mathbf{R}_{k+1} + t\mathbf{Q}_k\mathbf{P}_k.
\end{equation}
Введем обозначение $W_k \equiv (t \mathbf{Q}_k \mathbf{R}_{k+1} )(A) \circ R_0$. 
Тогда выражение \eqref{eq:W_k_poly} можно записать в матричном виде:
\begin{equation}
    \label{eq:W_k_matr}
    W_k = AS_k + AP_k\beta_k.
\end{equation}
Тогда выражение \eqref{eq:ort_beta_poly} можно представить в виде: 
\begin{equation}
    \label{eq:ort_beta_matr}
    \tilde{R}_0^*W_k = 0.
\end{equation}
Аналогично получаем соотношения реортогонализации для \eqref{eq:ort_beta_matr}:
\begin{equation*}
    W_k^r = W_k + AP_k\beta_k^r
\end{equation*}
Поправка $\beta_k$ определяется уравнением:
\begin{equation*}
    (\tilde{R}_0^*AP_k)\beta_k^r = -\tilde{R}_0^*W_k
\end{equation*}
Следующим шагом получим формулу для вычисления $P_{k+1}$ с учетом введённых обозначений
\begin{align*}
    \mathbf{Q}_{k+1}\mathbf{P}_{k+1} &= \mathbf{Q}_{k+1}(\mathbf{R}_{k+1} + \mathbf{P}_{k}\beta_k) = \\
                                     &= \mathbf{Q}_{k}\mathbf{R}_{k+1} - \omega_kt\mathbf{Q}_{k}\mathbf{R}_{k+1} + (1-\omega_kt)\mathbf{Q}_{k}\mathbf{P}_{k}\beta_k = \\
                                     &= \mathbf{Q}_{k} \mathbf{R}_{k+1} - \omega_kt\mathbf{Q}_{k}\mathbf{P}_{k+1} + \mathbf{Q}_{k} \mathbf{P}_{k} \beta_k
\end{align*}
В матричном виде это выражение записывавется как:
\begin{equation*}
    P_{k+1} = S_{k+1} + P_k \beta_k - \omega_k W_k
\end{equation*}

Для дополнительной минимизации нормы невязки поддерживается следующее соотношение:
$$ \langle AS_k,R_{k+1}\rangle_F=0 $$
Для этого выражения также можно выписать процедуру реортогонализации:
\begin{equation*}
    R_{k+1}^r = R_{k+1} - \omega_k^r T_k
\end{equation*}
\begin{equation*}
    \omega_k^r = \frac{\langle R_{k+1}, T_k \rangle_F}{\langle T_{k+1}, T_k \rangle_F}
\end{equation*}

\subsection{Ортогонализация векторов направлений и проверочных невязок}

$P_k = Q_PR_P$\\
\par $\tilde{R}_0$\\
Вспомним вид алгоритма [GUENNOUNI], в блоке Алгоритм \ref{alg:BCGSTAB_red} красным отмечены
все места где используется блок векторов направлений $P_k$. Легко видеть, что он везде
входит в алгоритм вместе матрицей коэффициентов ($\alpha_k$ и $\beta_k$). Так что если сделать замену
$P_k \gets P_k U$, где $U$ - $s \times s$ матрица, то изменятся лишь сами матрицы коэффициентов,
в то время как сами выражения в алгоритме останутся неизменными.       
\begin{algorithm}
    \caption{Блочные стабилизирвоанные бисопряженные градиенты}\label{alg:BCGSTAB_red}
    \begin{algorithmic}
    \State $X_0$ - начальное приближение
    \State $R_0 = B - AX_0$
    \State $P_0 = R_0$
    \State $\tilde{R}_0$ - произвольная $N \times s$ матрица
    \For {$k = 0, 1, 2, ...$}  
        \State решить $\tilde{R}_0^TA {\color{red}P_k\alpha_k} = \tilde{R}_0^TR_k$
        \State $S_k=R_k-A{\color{red}P_k\alpha_k}$
        \State $T_k=AS_k$
        \State $\omega_k = \frac{\langle T_k,S_k\rangle_F}{\langle T_k, T_k\rangle_F}$
        \State $X_{k+1}=X_k + {\color{red}P_k\alpha_k} + \omega_kS_k$
        \State $R_{k+1} = S_k - \omega_k T_k$
        \State решить $\tilde{R}_0^TA{\color{red}P_k\beta_k} = -\tilde{R}_0^TT_k$
        \State $P_{k+1}=R_{k+1} + {\color{red}P_k\beta_k} - \omega_k A{\color{red}P_k\beta_k}$
    \EndFor  
    \end{algorithmic}
\end{algorithm}
Так что можно попробовать подобрать такую $U$, чтобы вычисления стали более устойчивыми. Например, можно 
сделать $QR$-разложение матрицы $P_k$:
$$P_k = Q_PR_P,$$
и в качестве $U$ взять $R_k^{-1}$. Такой выбор $U$ повлечет ортогонализацию $P_k$, что должно улучшить стабильность
операций проектирования на вектора направлений.
\par Как указано в алгоритме \ref{alg:BCGSTAB_red}, $\tilde{R}_0$ - произвольная матрица, обычно ее выбираюсь равной
$R_0$. Аналогично для улучшения стабильности предлагается сделать $QR$-разложение матрицы
$R_0$:
$$R_0 = Q_R R_R,$$
и сделать замену $\tilde{R}_0 \gets Q_R$.

\subsection{Выбор правых частей}
% график с сингулярными числами
% \par rrqr
Алгоритм перестает сходиться, если блок невязок становится почти вырожденным, поэтому предлагается
на этапе инициализации алгоритма сделать RRQR-разложение блока правых частей, рассмотреть получившуюся
перестановку, и выбрать несколько правых частей с номерами, соответствующим первым номерам в перестановке.
Благодаря такому выбору рассматривается более линейно-независимый набор столбцов, что положительно сказывается
на сходимости.    

\subsection{Проблемы}
\par Основным недостатком блочного метода бисопряженных градиентов [GUENNOUNI] является то,
что $\omega_k$ является скаляром. Была большая надежда на то, что получится обобщить метод
на случай, когда $\omega_k$ - произвольная $s \times s$ матрица, но в ходе исследования выяснилось,
что это невозможно из-за некоммутативности матричнозначных полиномов, которая не позволяет получить короткие итерационные формулы.
\par возможные брейкдауны [GUENNOUNI]
\par В данном алгоритме возможны брейкдауны, в случаях, когда матрица $\tilde{R}_0AP_k$ становится
вырожденной. В такой ситуации авторы [GUENNOUNI] предлагают провести рестарт с другой $\tilde{R}_0$.
Также возможны стагнации, если $\omega_k \approx 0$. {\color{red} А что делать то в этом случае?}

\newpage
