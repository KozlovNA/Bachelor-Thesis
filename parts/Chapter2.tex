\section{Крыловские методы решения систем уравнений}
\label{sec:Chapter2} \index{Chapter2}

Здесь надо рассмотреть все существующие решения поставленной задачи, но не
просто пересказать, в чем там дело, а оценить степень их соответствия тем
ограничениям, которые были сформулированы в постановке задачи.

\subsection{Метод сопряженных градиентов}
[YOUSEF SAAD ITERATIVE METHODS FOR SPARSE LINEAR SYSTEMS SECOND EDITION]
\subsection{Метод бисопряженных градиентов}
[YOUSEF SAAD ITERATIVE METHODS FOR SPARSE LINEAR SYSTEMS SECOND EDITION]
\subsection{Метод стабилизированных бисопряженных градиентов}
\cite{doi:10.1137/0913035}
\subsection{Метод блочных сопряженных градиентов}
\cite{OLEARY1980293}
\subsection{Метод блочных бисопряженных градиентов}
\cite{OLEARY1980293}
\subsection{Метод блочных стабилизированных бисопряженных градиентов}
\cite{elGuennouni2003}
\subsubsection{Матричнозначные полиномы}
 \par Проблемы со сходимостью метода из \cite{elGuennouni2003}, демонстрация в 4 главе, решение проблемы в 3 главе.
 \subsubsection{Алгоритм}


\newpage
