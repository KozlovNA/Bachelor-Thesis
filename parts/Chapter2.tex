\section{Обзор существующих решений}
\label{sec:Chapter2} \index{Chapter2}

Здесь надо рассмотреть все существующие решения поставленной задачи, но не
просто пересказать, в чем там дело, а оценить степень их соответствия тем
ограничениям, которые были сформулированы в постановке задачи.

\subsection{Метод сопряженных градиентов}
[YOUSEF SAAD ITERATIVE METHODS FOR SPARSE LINEAR SYSTEMS SECOND EDITION]
\subsection{Метод бисопряженных градиентов}
[YOUSEF SAAD ITERATIVE METHODS FOR SPARSE LINEAR SYSTEMS SECOND EDITION]
\subsection{Метод стабилизированных бисопряженных градиентов}
[VAN DER VORST BI-CGSTAB: A FAST AND SMOOTHLY CONVEGRING VARIANT OF BI-CG FOR THE SOLUTION OF NONSYMMETRIC LINEAR SYSTEMS]
\subsection{Метод блочных сопряженных градиентов}
[DIANNE P. O'LEARY THE BLOCK CONJUGATE GRADIENT ALGORYTHM AND RELATED METHODS]
\subsection{Метод блочных бисопряженных градиентов}
[DIANNE P. O'LEARY THE BLOCK CONJUGATE GRADIENT ALGORYTHM AND RELATED METHODS]
\subsection{Метод блочных стабилизированных бисопряженных градиентов}
[GUENNOUNI A BLOCK VERSION OF BCGSTAB FOR LINEAR SYSTEMS WITH MULTIPLE RIGHT-HAND SIDES ]
\subsubsection{Матричнозначные полиномы}
 \par Проблемы со сходимостью метода из [GUENNOUNI], демонстрация в 4 главе, решение проблемы в 3 главе.
 \subsubsection{Алгоритм}


\newpage
