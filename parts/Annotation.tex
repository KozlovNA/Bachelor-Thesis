\begin{abstract}

    \begin{center}
        \large{Блочные методы типа бисопряжённых градиентов} \\
    \large\textit{Козлов Николай Андреевич} \\[1 cm]

В выпускной квалификационной работе исследуются блочные методы типа бисопряжённых 
градиентов для решения больших разреженных систем линейных уравнений с множеством
правых частей $AX=B$. Основное внимание уделяется стабилизированному методу бисопряженных
градиентов и его блочным аналогам, а также симметричному блочному методу квазиминимальных
невязок. Цель работы — повышение устойчивости и скорости сходимости алгоритмов на реальных задачах,
в частности - на задаче электромагнитного рассеяния на миндалевидном теле, дискретизированной 
с помощью метода RWG. 
\vfill

\textbf{Abstract} \\[1 cm]
\large{Block Krylov space methods alike biconjugate gradient method} \\
\large\textit{Kozlov Nikolai Andreevich}
In this graduation thesis, block biconjugate gradient-type methods are investigated 
for solving large sparse systems of linear equations with multiple right-hand sides 
$AX=B$. Primary focus is given to the stabilized biconjugate gradient method (BiCGStab)
and its block analogs, as well as the block symmetric quasi-minimal residual method. 
The work aims to enhance the stability and convergence rate of these algorithms for
 practical problems, specifically applied to the problem of electromagnetic scattering
from an almond-shaped body discretized using the RWG method.
\end{center}

\end{abstract}
\newpage